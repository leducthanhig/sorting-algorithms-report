\section{Giới thiệu}

Đồ án này được thực hiện trong khuôn khổ môn học \textit{Cấu trúc dữ liệu 
và Giải thuật} tại Khoa Công nghệ Thông tin, Đại học Khoa học Tự nhiên, 
ĐHQG TP.HCM. Đây là một môn học cốt lõi, giúp sinh viên nắm vững các 
khái niệm cơ bản về giải thuật, phân tích độ phức tạp và ứng dụng 
thực tế trong lập trình. Thông qua đồ án này, sinh viên sẽ có cơ hội 
thực hành sâu hơn về \textit{các thuật toán sắp xếp} – một trong những chủ đề 
quan trọng nhất trong lĩnh vực giải thuật. Mục tiêu chính là giúp 
sinh viên hiểu rõ cách các thuật toán sắp xếp hoạt động, các trường hợp 
ứng dụng cụ thể, cũng như cách đánh giá hiệu quả của từng thuật toán 
dựa trên các tiêu chí như \textit{thời gian thực thi} và \textit{số phép so sánh}.
\\\\
Đồ án tập trung vào việc khảo sát và phân tích \textit{các thuật toán sắp 
xếp}, từ các thuật toán cơ bản như \textbf{Selection Sort}, \textbf{Bubble 
Sort}, \textbf{Shaker Sort}, cũng như thuật toán \textbf{Insertion Sort} 
và thuật toán biến thể của nó là \textbf{Shell Sort}, đến các thuật toán 
nâng cao như \textbf{Merge Sort}, \textbf{Quick Sort}, và \textbf{Heap 
Sort}. Ngoài ra, một số thuật toán ít phổ biến và nâng cao hơn như 
\textbf{Counting Sort}, \textbf{Radix Sort} và \textbf{Flash Sort} cũng 
được đưa vào nghiên cứu để mở rộng kiến thức của sinh viên. Để đảm bảo 
tính toàn diện, đồ án yêu cầu thực hiện các bước: lý thuyết thuật toán, 
cài đặt thuật toán, thực nghiệm trên các loại dữ liệu đầu vào khác nhau, 
đo lường hiệu suất và phân tích kết quả thông qua các biểu đồ và bảng 
biểu rồi đưa ra nhận xét riêng cho từng kiểu dữ liệu, kích thước đầu vào 
với các thuật toán tương ứng. Đồng thời, đồ án cũng đưa ra các nhận xét 
tổng thể mang tính thiết thực, tóm gọn lại những gì đã nêu trên.
\\\\
Phần lý thuyết cung cấp nền tảng cơ bản về ý tưởng, cách hoạt động và độ 
phức tạp của các thuật toán được chọn cùng đưa ra các ví dụ đi kèm. Phần 
cài đặt thuật toán trình bày mã nguồn C++ và giải thích chi tiết cách 
triển khai từng thuật toán. Phần thực nghiệm và phân tích bao gồm các 
kết quả đo lường (về thời gian chạy, số phép so sánh) rồi so sánh hiệu 
suất của các thuật toán trong các trường hợp đầu vào khác nhau. Cuối cùng, 
phần kết luận đưa ra nhận xét tổng quan về hiệu quả của các thuật toán, 
các trường hợp sử dụng tối ưu và bài học rút ra từ đồ án. Đây là một bước 
đệm quan trọng giúp sinh viên phát triển tư duy thuật toán cũng như kỹ 
năng lập trình, chuẩn bị cho các môn học nâng cao và các dự án thực tế 
trong tương lai. Đồ án này không chỉ giúp sinh viên nắm vững kiến thức 
chuyên môn mà còn rèn luyện khả năng làm việc nhóm, giải quyết vấn đề và 
trình bày kết quả một cách chuyên nghiệp. 