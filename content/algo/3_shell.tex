\subsection{Shell Sort}

Là một thuật toán sắp xếp dựa trên sắp xếp chèn (Insertion Sort), được 
đặt theo tên của người phát minh, Donald Shell. Thuật toán này cải tiến 
sắp xếp chèn bằng cách cho phép so sánh và hoán đổi các phần tử cách xa 
nhau trước khi thu hẹp khoảng cách lại. Điều này giúp giảm thiểu số lần 
hoán đổi và dịch chuyển các phần tử. Đây là một thuật toán không ổn định, 
phần tử bằng nhau có thể bị đổi chỗ do hoán đổi các nhóm cách xa nhau.

\subsubsection{Ý tưởng}

\begin{enumerate}
    \item Thay vì sắp xếp toàn bộ danh sách một cách tuần tự, Shell Sort 
    chia danh sách thành các danh sách con với khoảng cách giữa các phần 
    tử được xác định bởi một giá trị gọi là khoảng cách (interval).
    \item Thuật toán sử dụng sắp xếp chèn để sắp xếp các phần tử trong 
    từng danh sách con này.
    \item Sau mỗi vòng lặp, khoảng cách được thu nhỏ dần, cuối cùng trở 
    về 1, lúc này danh sách được sắp xếp hoàn chỉnh.
\end{enumerate}

\subsubsection{Mã giả}

\begin{algorithm}[H]
\caption{Shell Sort}
\SetKwFunction{ShellSort}{ShellSort}
\SetKwProg{Fn}{Function}{:}{}
\Fn{\ShellSort{a\KwSty{[]}, n}}{
    // Bước 1: Khởi tạo khoảng cách \\
    $gap \gets n / 2$ \\
    \While{$gap > 0$}{
        // Bước 2: Sắp xếp chèn với khoảng cách gap \\
        \For{$i \gets gap$ \KwTo $n - 1$}{
            $temp \gets a[i]$ \\
            $j \gets i$ \\
            // Dịch chuyển các phần tử lớn hơn temp lên một khoảng cách \\
            \While{$j \geq gap$ \KwSty{and} $a[j - gap] > temp$}{
                $a[j] \gets a[j - gap]$ \\
                $j \gets j - gap$ \\
            }
            // Đặt temp vào đúng vị trí \\
            $a[j] \gets temp$ \\
        }
        // Bước 3: Giảm khoảng cách \\
        $gap \gets gap / 2$ \\
    }
}
\textbf{end function}
\end{algorithm}

\subsubsection{Ví dụ}

\subsubsection{Độ phức tạp thuật toán}

\begin{itemize}
    \item Độ phức tạp thời gian
    \begin{itemize}[label=$\circ$]
        \item Trường hợp tốt nhất: $O\left(n\log{n}\right)$, khi dữ liệu gần như đã được sắp xếp.
        \item Trường hợp xấu nhất: $O\left(n^2\right)$.
        \item Trường hợp trung bình: $O\left(n^{3/2}\right)$.
    \end{itemize}
    \item Độ phức tạp không gian: $O\left(1\right)$.
\end{itemize}