\subsection{Insertion Sort}

%Là thuật toán sắp xếp dựa trên phương pháp chèn trực tiếp. Thuật toán hoạt động hiệu quả trên mảng nhỏ hoặc mảng gần như đã sắp xếp.

\subsubsection{Ý tưởng}

Thuật toán này chia mảng thành hai phần: phần đã sắp xếp và 
phần chưa sắp xếp. Với mỗi phần tử trong mảng chưa sắp xếp, thuật toán chèn nó vào vị trí thích hợp trong phần đã sắp xếp.

\subsubsection{Mã giả}

\begin{algorithm}[H]
\caption{Insertion Sort}
\SetKwFunction{InsertionSort}{InsertionSort}
\SetKwProg{Fn}{Function}{:}{}
\Fn{\InsertionSort{a\KwSty{[]}, n}}{
    \For{$i \gets 1$ \KwTo $n - 1$}{
        $key \gets a[i]$ \\
        $j \gets i - 1$ \\
        \While{$j \geq 0$ \KwSty{and} $a[j] > key$}{
            // Di chuyển phần tử lớn hơn sang phải \\
            $a[j + 1] \gets a[j]$ \\
            $j \gets j - 1$ \\
        }
        // Chèn key vào đúng vị trí \\
        $a[j + 1] \gets key$
    }
}
\textbf{end function}
\end{algorithm}

\subsubsection{Ví dụ}

\begin{tikzpicture}[node distance=0cm, font=\sffamily, every node/.style={minimum width=1cm, minimum height=1cm, outer sep=0pt, anchor = west}, line join=miter, line cap=rect]
	
	% Trạng thái ban đầu
	\node[font=\rmfamily] at (0, 0) {Giả sử ta có mảng ban đầu với $n=6$ như sau:};
	\node[draw, fill=white] at (9, 0) {6};
	\node[draw, fill=white] at (10, 0) {1};
	\node[draw, fill=white] at (11, 0) {4};
	\node[draw, fill=white] at (12, 0) {8};
	\node[draw, fill=white] at (13, 0) {2};
	\node[draw, fill=white] at (14, 0) {15};
	
	% Trạng thái 1
	\node[font=\rmfamily] at (0, -1) {{\bfseries Bước 1:} Bắt đầu từ vị trí $i=1$, chèn 1 vào đúng vị trí của phần mảng đã sắp xếp (chỉ có 6).};
	\node[draw, fill=white] at (1, -2.5) {6};
	\node[draw, fill=yellow] at (2, -2.5) {1};
	\node[draw, fill=white] at (3, -2.5) {4};
	\node[draw, fill=white] at (4, -2.5) {8};
	\node[draw, fill=white] at (5, -2.5) {2};
	\node[draw, fill=white] at (6, -2.5) {15};
	\draw[line width=0.7mm, shorten <=0pt] (2.5, -2) -- (2.5, -1.5);
	\draw[line width=0.7mm] (2.5, -1.5) -- (1, -1.5);
	\draw[->, line width=0.7mm, shorten >=0pt] (1, -1.5) -- (1,-2);
	
	\draw[->, line width=0.7mm] (8, -2.5) -- (9.5, -2.5);
	
	\node[draw, fill=blue] at (10.5, -2.5) {1};
	\node[draw, fill=blue] at (11.5, -2.5) {6};
	\node[draw, fill=white] at (12.5, -2.5) {4};
	\node[draw, fill=white] at (13.5, -2.5) {8};
	\node[draw, fill=white] at (14.5, -2.5) {2};
	\node[draw, fill=white] at (15.5, -2.5) {15};
	
	% Trạng thái 2
	\node[font=\rmfamily] at (0, -4) {{\bfseries Bước 2:} Với vị trí $i=2$, chèn 4 vào đúng vị trí của phần mảng đã sắp xếp (gồm 1, 6).};
	\node[draw, fill=white] at (1, -5.5) {1};
	\node[draw, fill=white] at (2, -5.5) {6};
	\node[draw, fill=yellow] at (3, -5.5) {4};
	\node[draw, fill=white] at (4, -5.5) {8};
	\node[draw, fill=white] at (5, -5.5) {2};
	\node[draw, fill=white] at (6, -5.5) {15};
	\draw[line width=0.7mm, shorten <=0pt] (3.5, -5) -- (3.5, -4.5);
	\draw[line width=0.7mm] (3.5, -4.5) -- (2, -4.5);
	\draw[->, line width=0.7mm, shorten >=0pt] (2, -4.5) -- (2,-5);
	
	\draw[->, line width=0.7mm] (8, -5.5) -- (9.5, -5.5);
	
	\node[draw, fill=blue] at (10.5, -5.5) {1};
	\node[draw, fill=blue] at (11.5, -5.5) {4};
	\node[draw, fill=blue] at (12.5, -5.5) {6};
	\node[draw, fill=white] at (13.5, -5.5) {8};
	\node[draw, fill=white] at (14.5, -5.5) {2};
	\node[draw, fill=white] at (15.5, -5.5) {15};
		
\end{tikzpicture}

\pagebreak

\begin{tikzpicture}[node distance=0cm, font=\sffamily, every node/.style={minimum width=1cm, minimum height=1cm, outer sep=0pt, anchor = west}, line join=miter, line cap=rect]
	
	% Trạng thái 3
	\node[font=\rmfamily, text width=18cm] at (0, 0) {{\bfseries Bước 3:} Với $i=3$, không thay đổi vị trí của 8 vì nó lớn hơn tất cả phần tử trong phần mảng \par đã sắp xếp.};
	\node[draw, fill=white] at (1, -1.5) {1};
	\node[draw, fill=white] at (2, -1.5) {4};
	\node[draw, fill=white] at (3, -1.5) {6};
	\node[draw, fill=yellow] at (4, -1.5) {8};
	\node[draw, fill=white] at (5, -1.5) {2};
	\node[draw, fill=white] at (6, -1.5) {15};
	\draw[line width=0.7mm, shorten <=0pt] (4.5, -1) -- (4.5, -0.5);
	\draw[line width=0.7mm] (4.5, -0.5) -- (4, -0.5);
	\draw[->, line width=0.7mm, shorten >=0pt] (4, -0.5) -- (4, -1);
	
	\draw[->, line width=0.7mm] (8, -1.5) -- (9.5, -1.5);
	
	\node[draw, fill=blue] at (10.5, -1.5) {1};
	\node[draw, fill=blue] at (11.5, -1.5) {4};
	\node[draw, fill=blue] at (12.5, -1.5) {6};
	\node[draw, fill=blue] at (13.5, -1.5) {8};
	\node[draw, fill=white] at (14.5, -1.5) {2};
	\node[draw, fill=white] at (15.5, -1.5) {15};
	
	% Trạng thái 4
	\node[font=\rmfamily, text width=20cm] at (0, -3) {{\bfseries Bước 4:} Tương tự, với $i=4$, chèn 2 vào đúng vị trí.};
	\node[draw, fill=white] at (1, -4.5) {1};
	\node[draw, fill=white] at (2, -4.5) {4};
	\node[draw, fill=white] at (3, -4.5) {6};
	\node[draw, fill=white] at (4, -4.5) {8};
	\node[draw, fill=yellow] at (5, -4.5) {2};
	\node[draw, fill=white] at (6, -4.5) {15};
	\draw[line width=0.7mm, shorten <=0pt] (5.5, -4) -- (5.5, -3.5);
	\draw[line width=0.7mm] (5.5, -3.5) -- (2, -3.5);
	\draw[->, line width=0.7mm, shorten >=0pt] (2, -3.5) -- (2, -4);
	
	\draw[->, line width=0.7mm] (8, -4.5) -- (9.5, -4.5);
	
	\node[draw, fill=blue] at (10.5, -4.5) {1};
	\node[draw, fill=blue] at (11.5, -4.5) {2};
	\node[draw, fill=blue] at (12.5, -4.5) {4};
	\node[draw, fill=blue] at (13.5, -4.5) {6};
	\node[draw, fill=blue] at (14.5, -4.5) {8};
	\node[draw, fill=white] at (15.5, -4.5) {15};
	
	% Trạng thái 5
	\node[font=\rmfamily, text width=20cm] at (0, -6) {{\bfseries Bước 5:} Cuối cùng, sau khi chèn 15 vào đúng vị trí, ta thu được mảng đã sắp xếp.};
	\node[draw, fill=white] at (1, -7.5) {1};
	\node[draw, fill=white] at (2, -7.5) {4};
	\node[draw, fill=white] at (3, -7.5) {6};
	\node[draw, fill=white] at (4, -7.5) {8};
	\node[draw, fill=white] at (5, -7.5) {2};
	\node[draw, fill=yellow] at (6, -7.5) {15};
	\draw[line width=0.7mm, shorten <=0pt] (6.5, -7) -- (6.5, -6.5);
	\draw[line width=0.7mm] (6.5, -6.5) -- (6, -6.5);
	\draw[->, line width=0.7mm, shorten >=0pt] (6, -6.5) -- (6, -7);
	
	\draw[->, line width=0.7mm] (8, -7.5) -- (9.5, -7.5);
	
	\node[draw, fill=blue] at (10.5, -7.5) {1};
	\node[draw, fill=blue] at (11.5, -7.5) {2};
	\node[draw, fill=blue] at (12.5, -7.5) {4};
	\node[draw, fill=blue] at (13.5, -7.5) {6};
	\node[draw, fill=blue] at (14.5, -7.5) {8};
	\node[draw, fill=blue] at (15.5, -7.5) {15};
	
\end{tikzpicture}

\subsubsection{Độ phức tạp thuật toán}

\begin{itemize}
    \item Độ phức tạp thời gian
    \begin{itemize}[label=$\circ$]
        \item Trường hợp tốt nhất: $O\left(n\right)$ khi mảng đã được sắp xếp.
        \item Trường hợp xấu nhất: $O\left(n^2\right)$ khi mảng sắp xếp 
        ngược, cần dịch chuyển toàn bộ phần tử cho mỗi lần chèn.
        \item Trường hợp trung bình: $O\left(n^2\right)$ do phải thực 
        hiện n lần chèn và mỗi lần chèn có thể phải dịch chuyển trung 
        bình n/2 phần tử.
    \end{itemize}
    \item Độ phức tạp không gian: $O\left(1\right)$ vì là thuật toán 
    sắp xếp tại chỗ (in-place), không sử dụng bộ nhớ phụ ngoài biến tạm.
\end{itemize}

\subsubsection{Các cải tiến của thuật toán}

\begin{enumerate}
    \item Binary Insertion Sort
    \begin{itemize}
        \item Ý tưởng: Thay vì sử dụng tìm kiếm tuyến tính để tìm vị trí 
        chèn phù hợp trong mảng đã sắp xếp thì ta có thể sử dụng thuật 
        toán làm việc được trên mảng đã sắp xếp để có thể tối ưu việc 
        tìm vị trí phù hợp để chèn. Đó chính là Binary Search để có 
        thể tìm kiếm tốt hơn với độ phức tạp của thuật toán là 
        $O\left(\log{n}\right)$.
        \item Ví dụ: Trong mảng $A=\left[1,4,5,10,15,20,9,21\right]$. 
        Khi xét vị trí $i=7$, ta cần tìm vị trí phụ hợp để chèn 9 
        vào thì có thể dùng binary search trong mảng đã sắp xếp trước 
        đó $\left(\left[1,4,5,10,15,20\right]\right)$ để tìm vị trí thích hợp.
    \end{itemize}
    \item Library Sort (còn gọi là gapped insertion sort)
    \begin{itemize}
        \item Ý tưởng: Thuật toán này có thể sử dụng các khoảng trống 
        (gaps) trong mảng để chèn phần tử, giúp hạn chế việc dịch 
        chuyển quá nhiều khi chèn giúp tối ưu trong việc xử lí khi 
        mảng có rất nhiều phần tử. Khi khoảng trống hết, mảng sẽ được 
        mở rộng và phân bổ lại khoảng trống mới. Với độ phức tạp thời 
        gian trung bình của Library Sort là $O\left(n\log{n}\right)$.
        \item Ví dụ: Với mảng $A=\left[3,4,5,8,1,2\right]$. Sau khi 
        tạo mảng với khoảng trống và chèn 3, 4, 5, 8 thì mảng 
        như sau $\left[3,\_,4,5,8\right]$.
        \begin{itemize}[label=$\circ$]
            \item Như vậy, khi chèn 1 vào đúng vị trí thì sẽ sắp xếp 
            lại và điền vào khoảng trống $\left[1,3,4,5,8\right]$.
            \item Tiếp theo, chèn 2, mảng không còn vị trí nào sau 
            phần tử 3 thì khi chèn 2 thì lúc này khoảng trống sẽ hết 
            thì mảng sẽ được mở rộng và phân bổ lại khoảng trống mới. 
            Mảng lúc này sẽ là $\left[1,\_,\_,3,4,5,8\right]$ khi đó 
            khi chèn 2 vào thì sẽ sử dụng khoảng trống mới mở rộng 
            $\left[1,2,\_,3,4,5,8\right]$.
        \end{itemize}
    \end{itemize}
\end{enumerate}