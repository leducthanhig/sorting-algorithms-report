\subsection{Merge Sort}

\subsubsection{Ý tưởng}

Thuật toán sắp xếp trộn sử dụng phương pháp chia để trị (divide-and-conquer). Đầu tiên, chia mảng $a[1..n]$ thành 2 mảng con có kích thước $n/2$ là $a[1..mid]$ và $a[mid + 1..n]$, sắp xếp và trộn hai mảng này lại, ta thu được mảng $n/2$ phần tử đã được sắp xếp. Để sắp xếp mảng $n/2$ phần tử ta lại tiếp tục chia thành hai mảng có $n/4$ phần tử, cứ như vậy gọi đệ quy cho đến khi mảng chỉ có 1 phần tử thì xem như là đã được sắp xếp. \cite{idea-merge} \cite[p.~241]{weiss2013}

\subsubsection{Mã giả}

\begin{algorithm}[H]
\caption{Merge Sort \cite{idea-merge} \cite{code-merge}}
\label{merge-sort}

\SetKwFunction{mergeTwoArrays}{mergeTwoArrays}
\SetKwProg{Fn}{Function}{:}{}
\Fn{\mergeTwoArrays {arr\KwSty{[ ]}, arr1\KwSty{[ ]}, n1, arr2\KwSty{[ ]}, n2}}{
    $i \gets 0, \hspace{0.2cm} j \gets 0$ \\
    \While{$i < n1$ \KwSty{and} $j < n2$}{
        \If{$arr1[i] < arr2[j]$}{ 
            Thêm $arr1[i]$ vào $arr$ \\
            $++i$ \\
        }
        \Else{
            Thêm $arr2[j]$ vào $arr$ \\
            $++j$ \\
        }
    }
    \While{$i < n1$}{
        Thêm $arr1[i]$ vào $arr$ \\
        $++i$ \\
    }
    \While{$j < n2$}{
        Thêm $arr2[j]$ vào $arr$ \\
        $++j$ \\
    }
}
\textbf{end function}

\SetKwFunction{MergeSort}{MergeSort}
\SetKwProg{Fn}{Function}{:}{}
\Fn{\MergeSort {arr\KwSty{[ ]}, left, right}}{
    \If{$left < right$}{ 
        Tạo mảng $a1 = arr[left..mid]$ có $n1$ phần tử \\
        Tạo mảng $a2 = arr[mid+1..right]$ có $n2$ phần tử \\
        \MergeSort{arr, left, mid} \\
        \MergeSort{arr, mid + 1, right} \\
        \mergeTwoArrays{arr, a1, n1, a2, n2}
    }
}
\textbf{end function}
\end{algorithm}

\subsubsection{Ví dụ}

\begin{tikzpicture}[node distance=0cm, font=\sffamily, every node/.style={minimum width=1cm, minimum height=1cm, outer sep=0pt, anchor = west}, line join=miter, line cap=rect]
	
	\node[font=\rmfamily] at (0, 0) {}; %coordinate
	
	% Trạng thái ban đầu
	\node[draw, fill=white] at (4, 0) {8};
	\node[draw, fill=white] at (5, 0) {3};
	\node[draw, fill=white] at (6, 0) {1};
	\node[draw, fill=white] at (7, 0) {7};
	\node[draw, fill=white] at (8, 0) {0};
	\node[draw, fill=white] at (9, 0) {10};
	\node[draw, fill=white] at (10, 0) {2};
	\draw[->, >=stealth, line width=0.5mm, shorten <=0pt] (7, -0.55) -- (4, -1.45);
	\draw[->, >=stealth, line width=0.5mm, shorten <=0pt] (7, -0.55) -- (10, -1.45);
	
	% Chia đôi lần 1
	\node[draw, fill=yellow] at (2.5, -2) {8};
	\node[draw, fill=yellow] at (3.5, -2) {3};
	\node[draw, fill=yellow] at (4.5, -2) {1};
	\node[draw, fill=yellow] at (8, -2) {7};
	\node[draw, fill=yellow] at (9, -2) {0};
	\node[draw, fill=yellow] at (10, -2) {10};
	\node[draw, fill=yellow] at (11, -2) {2};
	\draw[->, >=stealth, line width=0.5mm, shorten <=0pt] (3.5, -2.55) -- (2, -3.45);
	\draw[->, >=stealth, line width=0.5mm, shorten <=0pt] (3.5, -2.55) -- (5, -3.45);
	\draw[->, >=stealth, line width=0.5mm, shorten <=0pt] (10, -2.55) -- (8.5, -3.45);
	\draw[->, >=stealth, line width=0.5mm, shorten <=0pt] (10, -2.55) -- (12, -3.45);
	
	% Chia đôi lần 2
	\node[draw, fill=yellow] at (1.5, -4) {8};
	\node[draw, fill=yellow] at (4, -4) {3};
	\node[draw, fill=yellow] at (5, -4) {1};
	\node[draw, fill=yellow] at (7.5, -4) {7};
	\node[draw, fill=yellow] at (8.5, -4) {0};
	\node[draw, fill=yellow] at (11, -4) {10};
	\node[draw, fill=yellow] at (12, -4) {2};
	\draw[->, >=stealth, line width=0.5mm, shorten <=0pt] (2, -4.55) -- (2, -5.45);
	\draw[->, line width=0.5mm, shorten <=0pt] (5, -4.55) -- (4, -5.45);
	\draw[->, >=stealth, line width=0.5mm, shorten <=0pt] (5, -4.55) -- (6, -5.45);
	\draw[->, >=stealth, line width=0.5mm, shorten <=0pt] (8.5, -4.55) -- (7.5, -5.45);
	\draw[->, >=stealth, line width=0.5mm, shorten <=0pt] (8.5, -4.55) -- (9.5, -5.45);
	\draw[->, >=stealth, line width=0.5mm, shorten <=0pt] (12, -4.55) -- (11, -5.45);
	\draw[->, >=stealth, line width=0.5mm, shorten <=0pt] (12, -4.55) -- (13, -5.45);
	
	% Chia đôi lần 3
	\node[draw, fill=yellow] at (1.5, -6) {8};
	\node[draw, fill=yellow] at (3.5, -6) {3};
	\node[draw, fill=yellow] at (5.5, -6) {1};
	\node[draw, fill=yellow] at (7, -6) {7};
	\node[draw, fill=yellow] at (9, -6) {0};
	\node[draw, fill=yellow] at (10.5, -6) {10};
	\node[draw, fill=yellow] at (12.5, -6) {2};
	\draw[->, >=stealth, line width=0.5mm, shorten <=0pt] (2, -6.55) -- (2, -7.45);
	\draw[->, >=stealth, line width=0.5mm, shorten <=0pt] (4, -6.55) -- (5, -7.45);
	\draw[->, >=stealth, line width=0.5mm, shorten <=0pt] (6, -6.55) -- (5, -7.45);
	\draw[->, >=stealth, line width=0.5mm, shorten <=0pt] (7.5, -6.55) -- (8.5, -7.45);
	\draw[->, >=stealth, line width=0.5mm, shorten <=0pt] (9.5, -6.55) -- (8.5, -7.45);
	\draw[->, >=stealth, line width=0.5mm, shorten <=0pt] (11, -6.55) -- (12, -7.45);
	\draw[->, >=stealth, line width=0.5mm, shorten <=0pt] (13, -6.55) -- (12, -7.45);
	
	% Trộn lần 1
	\node[draw, fill=blue] at (1.5, -8) {8};
	\node[draw, fill=blue] at (4, -8) {1};
	\node[draw, fill=blue] at (5, -8) {3};
	\node[draw, fill=blue] at (7.5, -8) {0};
	\node[draw, fill=blue] at (8.5, -8) {7};
	\node[draw, fill=blue] at (11, -8) {2};
	\node[draw, fill=blue] at (12, -8) {10};
	\draw[->, >=stealth, line width=0.5mm, shorten <=0pt] (2, -8.55) -- (3.5, -9.45);
	\draw[->, >=stealth, line width=0.5mm, shorten <=0pt] (5, -8.55) -- (3.5, -9.45);
	\draw[->, >=stealth, line width=0.5mm, shorten <=0pt] (8.5, -8.55) -- (10, -9.45);
	\draw[->, >=stealth, line width=0.5mm, shorten <=0pt] (12, -8.55) -- (10, -9.45);
	
	% Trộn lần 2
	\node[draw, fill=blue] at (2.5, -10) {1};
	\node[draw, fill=blue] at (3.5, -10) {3};
	\node[draw, fill=blue] at (4.5, -10) {8};
	\node[draw, fill=blue] at (8, -10) {0};
	\node[draw, fill=blue] at (9, -10) {2};
	\node[draw, fill=blue] at (10, -10) {7};
	\node[draw, fill=blue] at (11, -10) {10};
	\draw[->, >=stealth, line width=0.5mm, shorten <=0pt] (4, -10.55) -- (7, -11.45);
	\draw[->, >=stealth, line width=0.5mm, shorten <=0pt]  (10, -10.55) -- (7, -11.45);
	
	%Trộn lần 3
	\node[draw, fill=blue] at (4, -12) {0};
	\node[draw, fill=blue] at (5, -12) {1};
	\node[draw, fill=blue] at (6, -12) {2};
	\node[draw, fill=blue] at (7, -12) {3};
	\node[draw, fill=blue] at (8, -12) {7};
	\node[draw, fill=blue] at (9, -12) {8};
	\node[draw, fill=blue] at (10, -12) {10};
	
	% Giai đoạn 1
	\draw[line width=0.3mm, shorten <=0pt]  (14, 0) -- (14.2, 0);
	\draw[line width=0.3mm, shorten <=0pt]  (14.2, 0) -- (14.2, -5.8);
	\draw[line width=0.3mm, shorten <=0pt]  (14.2, -5.8) -- (14, -5.8);
	\node[font=\rmfamily, text width=3cm] at (14.4, -2.8) {{\bfseries Giai đoạn 1:} \par Chia đôi mảng \par liên tục cho \par đến khi chỉ \par còn 1 phần tử};
	
	% Giai đoạn 2
	\draw[line width=0.3mm, shorten <=0pt]  (14, -6) -- (14.2, -6);
	\draw[line width=0.3mm, shorten <=0pt]  (14.2, -6) -- (14.2, -12);
	\draw[line width=0.3mm, shorten <=0pt]  (14.2, -12) -- (14, -12);
	\node[font=\rmfamily, text width=3cm] at (14.4, -8.8) {{\bfseries Giai đoạn 2:} \par Trộn các mảng \par đã có thứ tự};
\end{tikzpicture}

\subsubsection{Độ phức tạp thuật toán}

\begin{itemize}
    \item Độ phức tạp thời gian \cite[p.~243--244]{weiss2013} \\
    Mảng ban đầu có $n$ phần tử cần sắp xếp sẽ được chia đôi nhiều lần cho đến khi nó chỉ còn 1 phần tử, như vậy sẽ có $log_2{n}$ mức đệ quy. Ở mỗi mức đệ quy, ta duyệt qua các phần tử ở tất cả mảng con để thực hiện trộn nên độ phức tạp là $O\left(n\right)$. Như vậy độ phức tạp của thuật toán Merge Sort là $O\left(n\log{n}\right)$. Có thể thấy thuật toán không phụ thuộc vào phân bố ban đầu của dữ liệu, do đó ta có:
    \begin{itemize}[label=$\circ$]
        \item Trường hợp tốt nhất: $O\left(n\log{n}\right)$
        \item Trường hợp xấu nhất: $O\left(n\log{n}\right)$
        \item Trường hợp trung bình: $O\left(n\log{n}\right)$ 
    \end{itemize}
    
    \item Độ phức tạp không gian: $O\left(n\right)$, vì thuật toán cần tạo thêm không gian để chứa những phần tử từ 2 mảng con rồi mới trộn vào mảng ban đầu.
\end{itemize}