\subsection{Heap Sort}

\subsubsection{Ý tưởng \textnormal{\cite{idea-heap} \cite[p.~170]{cormen2022}}}

\begin{itemize}
    \item Xây dựng Max-Heap: Chuyển đổi mảng đầu vào thành một Max-Heap, 
    sao cho phần tử lớn nhất nằm ở gốc cây.
    \item Sắp xếp:
    \begin{enumerate}
        \item Hoán đổi phần tử gốc (lớn nhất) với phần tử cuối cùng 
        của mảng.
        \item Giảm kích thước của heap đi 1 và gọi hàm Heapify để 
        duy trì tính chất Max-Heap.
        \item Lặp lại quá trình này cho đến khi heap chỉ còn một phần tử. 
    \end{enumerate}
\end{itemize}

\subsubsection{Mã giả}

\begin{algorithm}[H]
\caption{Heap Sort \cite[p.~165,167,170]{cormen2022} \cite{code-heap}}
\SetKwFunction{Heapify}{heapify}
\SetKwFunction{HeapSort}{HeapSort}
\SetKwProg{Fn}{Function}{:}{}
\Fn{\Heapify{a\KwSty{[]}, n, i}}{
    $largest \gets i$ \\
    $left \gets 2 * i + 1$ \\
    $right \gets 2 * i + 2$ \\
    \If{$left < n$ \KwSty{and} $a[left] > a[largest]$}{
        $largest \gets left$ \\
    }
    \If{$right < n$ \KwSty{and} $a[right] > a[largest]$}{
        $largest \gets right$ \\
    }
    \If{$largest \neq i$}{
        // Đổi chỗ root với phần tử lớn nhất \\
        swap($a[i]$, $a[largest]$) \\
        // Đệ quy heapify trên node bị ảnh hưởng \\
        \Heapify{a, n, largest} \\
    }
}
\textbf{end function}

\Fn{\HeapSort{a\KwSty{[]}, n}}{
    // Bước 1: Xây dựng Max-Heap từ node trong cuối cùng \\
    \For{$i \gets \lfloor n / 2 \rfloor - 1$ \KwSty{downto} $0$}{
        \Heapify{a, n, i} \\
    }
    // Bước 2: Sắp xếp mảng bằng cách trích xuất các phần tử lớn nhất \\
    \For{$i \gets n - 1$ \KwSty{downto} $1$}{
        swap($a[0]$, $a[i]$) \\
        \Heapify{a, i, 0} \\
    }
}
\textbf{end function}
\end{algorithm}

\subsubsection{Ví dụ}

\begin{tikzpicture}[node distance=0cm, font=\sffamily, every node/.style={minimum width=1cm, minimum height=1cm, outer sep=0pt, anchor = west}, line join=miter, line cap=rect]
	
	% Trạng thái ban đầu
	\node[font=\rmfamily] at (0, 0) {Giả sử ta có mảng ban đầu với $n=6$ như sau:};
	\node[draw, fill=white] at (9, 0) {6};
	\node[draw, fill=white] at (10, 0) {1};
	\node[draw, fill=white] at (11, 0) {4};
	\node[draw, fill=white] at (12, 0) {8};
	\node[draw, fill=white] at (13, 0) {23};
	\node[draw, fill=white] at (14, 0) {2};
	
	% Build max-heap
	\node[font=\rmfamily] at (0, -1) {\bfseries Bước 1: Xây dựng max-heap};
	
	% Cặp 1
	\node[font=\rmfamily] at (0, -2) {$\bullet$ Bắt đầu với $i = \left\lfloor \dfrac{n}{2} \right\rfloor - 1 = 2$. So sánh 4 với 2, không xảy ra hoán vị.};
	\node[draw, fill=white] at (1, -3.5) {6};
	\node[draw, fill=white] at (2, -3.5) {1};
	\node[draw, fill=red!50] at (3, -3.5) {4};
	\node[draw, fill=white] at (4, -3.5) {8};
	\node[draw, fill=white] at (5, -3.5) {23};
	\node[draw, fill=yellow] at (6, -3.5) {2};
	
	% Cặp 2
	\node[font=\rmfamily] at (0, -5) {$\bullet$ Tiếp tục với $i = 1$. Lần lượt so sánh 1 với 8 và 23, hoán vị 1 với 23.};
	\node[draw, fill=white] at (1, -6) {6};
	\node[draw, fill=red!50] at (2, -6) {1};
	\node[draw, fill=white] at (3, -6) {4};
	\node[draw, fill=yellow] at (4, -6) {8};
	\node[draw, fill=yellow] at (5, -6) {23};
	\node[draw, fill=white] at (6, -6) {2};
	
	\draw[->, line width=0.5mm] (8, -6) -- (9, -6);
	
	\node[draw, fill=white] at (10, -6) {6};
	\node[draw, fill=red!50] at (11, -6) {23};
	\node[draw, fill=white] at (12, -6) {4};
	\node[draw, fill=yellow] at (13, -6) {8};
	\node[draw, fill=yellow] at (14, -6) {1};
	\node[draw, fill=white] at (15, -6) {2};
	
	% Cặp 3
	\node[font=\rmfamily] at (0, -7.5) {$\bullet$ Tiếp tục với $i = 0$. Lần lượt so sánh 6 với 23 và 4, hoán vị 6 với 23.};
	\node[draw, fill=red!50] at (1, -8.5) {6};
	\node[draw, fill=yellow] at (2, -8.5) {23};
	\node[draw, fill=yellow] at (3, -8.5) {4};
	\node[draw, fill=white] at (4, -8.5) {8};
	\node[draw, fill=white] at (5, -8.5) {1};
	\node[draw, fill=white] at (6, -8.5) {2};
	
	\draw[->, line width=0.5mm] (8, -8.5) -- (9, -8.5);
	
	\node[draw, fill=red!50] at (10, -8.5) {23};
	\node[draw, fill=yellow] at (11, -8.5) {6};
	\node[draw, fill=yellow] at (12, -8.5) {4};
	\node[draw, fill=white] at (13, -8.5) {8};
	\node[draw, fill=white] at (14, -8.5) {1};
	\node[draw, fill=white] at (15, -8.5) {2};
	
	% Cặp 3
	\node[font=\rmfamily] at (0, -10) {$\bullet$ Hiệu chỉnh lan truyền: Lần lượt so sánh 6 với 8 và 1, hoán vị 6 với 8.};
	\node[draw, fill=white] at (1, -11) {23};
	\node[draw, fill=red!50] at (2, -11) {6};
	\node[draw, fill=white] at (3, -11) {4};
	\node[draw, fill=yellow] at (4, -11) {8};
	\node[draw, fill=yellow] at (5, -11) {1};
	\node[draw, fill=white] at (6, -11) {2};
	
	\draw[->, line width=0.5mm] (8, -11) -- (9, -11);
	
	\node[draw, fill=white] at (10, -11) {23};
	\node[draw, fill=red!50] at (11, -11) {8};
	\node[draw, fill=white] at (12, -11) {4};
	\node[draw, fill=yellow] at (13, -11) {6};
	\node[draw, fill=yellow] at (14, -11) {1};
	\node[draw, fill=white] at (15, -11) {2};
	
	% Sắp xếp
	\node[font=\rmfamily] at (0, -12.5) {\bfseries Bước 2: Sắp xếp mảng bằng cách lấy ra phần tử lớn nhất và heapify phần còn lại};
	
	% Phần tử 1
	\node[font=\rmfamily] at (0, -13.5) {$\bullet$ Hoán vị 23 với phần tử cuối cùng của heap, giảm kích thước heap xuống còn 5 và heapify lại.};
	\node[draw, fill=white] at (1, -14.5) {2};
	\node[draw, fill=white] at (2, -14.5) {8};
	\node[draw, fill=white] at (3, -14.5) {4};
	\node[draw, fill=white] at (4, -14.5) {6};
	\node[draw, fill=white] at (5, -14.5) {1};
	\node[draw, fill=blue] at (6, -14.5) {23};
	
	\draw[->, line width=0.5mm] (8, -14.5) -- (9, -14.5);
	
	\node[draw, fill=white] at (10, -14.5) {8};
	\node[draw, fill=white] at (11, -14.5) {6};
	\node[draw, fill=white] at (12, -14.5) {4};
	\node[draw, fill=white] at (13, -14.5) {2};
	\node[draw, fill=white] at (14, -14.5) {1};
	\node[draw, fill=blue] at (15, -14.5) {23};
	
	% Phần tử 2
	\node[font=\rmfamily] at (0, -16) {$\bullet$ Hoán vị 8 với phần tử cuối cùng của heap, giảm kích thước heap xuống còn 4 và heapify lại.};
	\node[draw, fill=white] at (1, -17) {1};
	\node[draw, fill=white] at (2, -17) {6};
	\node[draw, fill=white] at (3, -17) {4};
	\node[draw, fill=white] at (4, -17) {2};
	\node[draw, fill=blue] at (5, -17) {8};
	\node[draw, fill=blue] at (6, -17) {23};
	
	\draw[->, line width=0.5mm] (8, -17) -- (9, -17);
	
	\node[draw, fill=white] at (10, -17) {6};
	\node[draw, fill=white] at (11, -17) {2};
	\node[draw, fill=white] at (12, -17) {4};
	\node[draw, fill=white] at (13, -17) {1};
	\node[draw, fill=blue] at (14, -17) {8};
	\node[draw, fill=blue] at (15, -17) {23};
	
	% Phần tử 3
	\node[font=\rmfamily] at (0, -18.5) {$\bullet$ Hoán vị 6 với phần tử cuối cùng của heap, giảm kích thước heap xuống còn 3 và heapify lại.};
	\node[draw, fill=white] at (1, -19.5) {1};
	\node[draw, fill=white] at (2, -19.5) {2};
	\node[draw, fill=white] at (3, -19.5) {4};
	\node[draw, fill=blue] at (4, -19.5) {6};
	\node[draw, fill=blue] at (5, -19.5) {8};
	\node[draw, fill=blue] at (6, -19.5) {23};
	
	\draw[->, line width=0.5mm] (8, -19.5) -- (9, -19.5);
	
	\node[draw, fill=white] at (10, -19.5) {4};
	\node[draw, fill=white] at (11, -19.5) {2};
	\node[draw, fill=white] at (12, -19.5) {1};
	\node[draw, fill=blue] at (13, -19.5) {6};
	\node[draw, fill=blue] at (14, -19.5) {8};
	\node[draw, fill=blue] at (15, -19.5) {23};
	
	% Phần tử 4
	\node[font=\rmfamily] at (0, -21) {$\bullet$ Hoán vị 4 với phần tử cuối cùng của heap, giảm kích thước heap xuống còn 2 và heapify lại.};
	\node[draw, fill=white] at (1, -22) {1};
	\node[draw, fill=white] at (2, -22) {2};
	\node[draw, fill=blue] at (3, -22) {4};
	\node[draw, fill=blue] at (4, -22) {6};
	\node[draw, fill=blue] at (5, -22) {8};
	\node[draw, fill=blue] at (6, -22) {23};
	
	\draw[->, line width=0.5mm] (8, -22) -- (9, -22);
	
	\node[draw, fill=white] at (10, -22) {2};
	\node[draw, fill=white] at (11, -22) {1};
	\node[draw, fill=blue] at (12, -22) {4};
	\node[draw, fill=blue] at (13, -22) {6};
	\node[draw, fill=blue] at (14, -22) {8};
	\node[draw, fill=blue] at (15, -22) {23};
	
	% Phần tử 5
	\node[font=\rmfamily,  text width=10cm] at (0, -23.5) {$\bullet$ Hoán vị 2 với phần tử cuối cùng của heap, \par \hspace{0.35cm} ta thu được mảng đã sắp xếp:};
	\node[draw, fill=blue] at (9, -23.5) {1};
	\node[draw, fill=blue] at (10, -23.5) {2};
	\node[draw, fill=blue] at (11, -23.5) {4};
	\node[draw, fill=blue] at (12, -23.5) {6};
	\node[draw, fill=blue] at (13, -23.5) {8};
	\node[draw, fill=blue] at (14, -23.5) {23};
	
\end{tikzpicture}

\subsubsection{Độ phức tạp thuật toán}

\begin{itemize}
    \item Độ phức tạp thời gian \cite[p.~102]{hoang2008}
    \begin{itemize}[label=$\circ$]
        \item Trường hợp tốt nhất: $O(n\log{n})$ vì khi xây dựng heap 
        và thực hiện thao tác heapify vẫn cần $O(n \log n)$ do tính 
        chất của heap.
        \item Trường hợp xấu nhất: $O(n\log{n})$ trong trường hợp tệ 
        nhất (mảng hoàn toàn ngược hoặc có cấu trúc phức tạp) vẫn duy 
        trì $O(n \log n)$ vì mọi thao tác chính đều dựa trên việc 
        heapify từng phần tử.
        \item Trường hợp trung bình: $O(n\log{n})$ trường hợp trung bình 
        vẫn cần xây dựng heap và thực hiện $n-1$ lần thao tác Heapify.
    \end{itemize}
    \item Độ phức tạp không gian: $O(1)$ vì Heap Sort là thuật toán 
    sắp xếp tại chỗ (in-place), không sử dụng bộ nhớ phụ ngoài biến tạm.
\end{itemize}